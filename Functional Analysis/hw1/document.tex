
\documentclass[oneside, final, 11pt]{article}
\usepackage[utf8]{inputenc}
\usepackage[T2A]{fontenc}
\usepackage[russian]{babel}

\usepackage{natbib}
\usepackage{graphicx}
\usepackage{amsmath}
\usepackage{xcolor}
\usepackage{amssymb}
\usepackage{graphicx}
\usepackage[document]{ragged2e}
\usepackage[paper=a4paper, margin=2cm, bottom=2cm]{geometry}

\begin{document}
\begin{titlepage}
	
	
	\newgeometry{margin=1cm}
	
	\centerline{\large \bf МИНИСТЕРСТВО ОБРАЗОВАНИЯ РЕСПУБЛИКИ БЕЛАРУСЬ}
	\bigskip
	\bigskip
	\centerline{\large \bf БЕЛОРУССКИЙ ГОСУДАРСТВЕННЫЙ УНИВЕРСИТЕТ}
	\bigskip
	\bigskip
	\centerline{\large \bf ФАКУЛЬТЕТ ПРИКЛАДНОЙ МАТЕМАТИКИ И ИНФОРМАТИКИ}
	\vfill
	\vfill
	\vfill
	\centerline{\Large \bf ФУНКЦИОНАЛЬНЫЙ АНАЛИЗ}
	\bigskip
	\bigskip
	\vfill
	\begin{centering}
		{\large
			Домашняя работа \\
			студента 2 курса 2 группы \\}
	\end{centering}
	\centerline{\large \bf Царика Виталия Александровича}
	\vfill
	\vfill
	\hfill
	\begin{minipage}{0.25\textwidth}
		{\large{\bf Преподаватель} \\
			{\it Дайняк Виктор \\ Владимирович}}
	\end{minipage}
	\vfill
	\vfill
	\centerline{\Large \bf Минск 2019}
	
\end{titlepage}

\restoregeometry
\section{№1.8}
\begin{equation*}
    a=1, b=1, x(t) = \lambda \int_{-1}^1(t^2-1)sx(s) ds+t
\end{equation*}

  $F(x) = \lambda \int_{-1}^1(t^2-1)sx(s) ds+t $
 \begin{itemize} \item $C[-1,1]:$ \end{itemize} 
  ${max}_{t\in[-1,1]} |F(x)-F(y)| = {max}_{t\in[-1,1]}$ $\left|\int_{-1}^1(t^2-1)s(x(s)-y(s) ds\right| \leq $\\
 $ \leq |\lambda| \int_{-1}^1|s|*{max}_{s\in[-1,1]}|x(s)-y(s)| ds = \lambda$ ${max}_{s\in[-1,1]}|(x(s)-y(s)| = \lambda ||x(s)-y(s)||$ \\
 $ \alpha = |\lambda| < 1$ \\
  $\text{Пусть} \lambda = \frac{1}{14}$ \\
  $||x_n - a|| \leq \frac{\alpha^n}{1-\alpha}||x_0 - x_1|| $\\
  $x_0 = 0 $\\
  $x_1 = F(x_0) = t$ \\
  $\alpha = \frac{1}{14}$\\
 $ ||x_0 - x_1|| = {max}_{t\in[-1,1]}|t-0| = 1$ \\
 $ \left(\frac{1}{14}\right)^n\frac{14}{13} < 0.001$ \\
  $n > \left(\frac{\ln{\frac{13}{14}*0.001}}{\ln{14}}\right) \approx 2.645 $\\
  $n = 3$ \\
 $ x_2 = F(x_1) = F(t) = \frac{1}{14}(t^2-1)\int_{-1}^1 s^2 ds +t =$ $\frac{1}{14}(t^2-1) \left[\frac{s^3}{3}\right]_{-1}^1 +t =$ $\frac{1}{21}(t^2-1)+t$\\
  $x_3 = F(x_2) = F\left(\frac{1}{21}(t^2-1)+s\right) $= $\frac{1}{14}(t^2-1)\int_{-1}^1 s\left(\frac{1}{21}(s^2-1)+t\right) ds$ $= \frac{1}{21*14}(t^2-1)\left[\frac{s^4}{4}-\frac{s^2}{2}+21\frac{s^3}{3}\right]_{-1}^1 \Rightarrow$\\
  $x_3 = \frac{1}{21}(t^2-1)+t$\\
    $\text{Точное решение:}$\\
  $x(t) = \lambda(t^2-1) \int_{-1}^1sx(s) ds+t$ \\
  $c = \int_{-1}^1sx(s) ds \Rightarrow $\\
  $x(t) =  \lambda(t^2-1)c+t $\\
  $c = \int_{-1}^1s\left(\lambda(s^2-1)c+s\right)ds =$ $\left[c\lambda\frac{s^4}{4}-c\lambda\frac{s^2}{2}+\frac{s^3}{3}\right]_{-1}^1 = \frac{2}{3} \Rightarrow $\\
  $x(t)_ = \frac{1}{14}(t^2-1)\frac{2}{3}+t = \frac{1}{21}(t^2-1)+t$\\
  $||x-x_3|| = {max}_{t\in[-1,1]}\left(\frac{1}{21}(t^2-1)+t-\frac{1}{21}(t^2-1)-t\right)=0$\\
  \begin{itemize}
    \item $L_2[-1,1]:$ \\
  \end{itemize}
  $K(t,s) = \lambda(t^2-1)s $\\
  $\int_{-1}^1 \int_{-1}^1 \left|K(t,s)\right|^2 ds dt = \lambda^2 \int_{-1}^1 \int_{-1}^1 (t^2-1)^2 s^2 ds dt = \lambda^2$ $\int_{-1}^1 (t^2-1)^2 dt \int_{-1}^1 s^2 ds = \frac{2}{3}\lambda^2*\frac{16}{15} = \frac{32}{45}\lambda^2 < + \infty $\\
  $\text{F(x) является сжимающим, если }_{} \frac{4\sqrt{2}|\lambda|}{3\sqrt{3}} < 1 \Rightarrow |\lambda| <$ $\frac{3\sqrt{3}}{4\sqrt{2}} $\\
  $\text{Пусть }\lambda = \frac{1}{14}$ \\
  $_{}\frac{\left(\frac{4\sqrt{2}}{3\sqrt{3}}\frac{1}{14}\right)^n}{1-\frac{4\sqrt{2}}{3\sqrt{3}}\frac{1}{14}}$ $\left(\int_{-1}^1(x_1(t)-x_0(t))^2dt\right)^{1\over2} < 0.001 \Rightarrow$\\
  $_{}\frac{\left(\frac{2\sqrt{2}}{21\sqrt{3}}\frac{1}{14}\right)^n}{1-\frac{2\sqrt{2}}{21\sqrt{3}}\frac{1}{14}}*\sqrt{\frac{2}{3}} < 0.001 $\\
  $n > 2.486 $\\
  $n = 3 $\\

\section{№2.8}
\begin{flushleft}
   $ g(x) = 2x^2+8x-3=0$ \\
    $x = - \frac{2x^2-3}{8} $\\
    $F(x) = - \frac{2x^2-3}{8} $\\
    $\alpha = max_{x\in[a;b]}|F'(x)|$\\
    $F^{'}(x) = -\frac{4x}{8} = - \frac{x}{2}$\\
    $|F^{'}(x)| < 1, |x| < 2 $\\
    $\begin{cases}
    ||x_0 - F(x_0)|| \leq r(1-\alpha(r)) \\
    \alpha(r) < 1 \\
    \end{cases}$\\
    $\alpha(r) = max_{x\in[-r;r]}|F^{'}(x)| = \frac{r}{2},$\\
    $x_1 = F(x_0) = \frac{3}{8} $\\
   $ \begin{cases}
    \frac{3}{8} \leq r(1-\frac{r}{2})\\
    \frac{r}{2} < 1\\
    \end{cases}$\\
    $ Пусть  r=1 \Rightarrow  F - сжимающее,  \alpha \leq \frac{1}{2}$ \\
    $||x_n - \alpha|| \leq \frac{\alpha^n}{1-\alpha}||x_1 - x_0||\leq \left(\frac{1}{2}\right)^n*2*\frac{3}{8}\leq$ $\frac{1}{100}$\\
    $2^{-n} \leq 75^{-1}$ \\
    $n\ln{2} \geq \ln{75}$ \\
    $n \geq \frac{\ln{75}}{\ln{2}} \approx 6,229$ \\
    $n = 7 \Rightarrow x_7 \text{является приближённым решением уравнения с заданной точностью, равной 0,01}$\\
    $||x||_p = (\int_a^b |x(t)|^p dt)^{\frac{1}{p}}, p \geq 1 $\\
\end{flushleft}

\section{№3.8}
\begin{flushleft}
    $f(x)(t) = t \int_0^1 \frac{x(s)}{\sqrt[4]{s}} ds $\\
    $f: E \rightarrow E, E = L_2[0;1]$\\
    $||F(x)-F(y)||_{L_2[0;1]} = \left( \int_0^1 \left|t \int_0^1 \frac{x(s)-y(s)}{\sqrt[4]{s}} ds \right|^2 dt\right)^{1\over2}$ $\leq \left( \int_0^1 t^2 \left(\int_0^1 |x(s)-y(s)|^2 \frac{1}{\sqrt{s}} ds \right)dt\right)^{1\over2} \leq $\\
    $\leq \left( \int_0^1 t^2 \left(\int_0^1 |x(s)-y(s)|^2 {max}_{s\in[0;1]} s^{1\over2} ds \right)dt\right)^{1\over2} = \left(\int_0^1 t^2 dt \right)^{1\over2} ||x-y|| = \frac{1}{\sqrt{3}} ||x-y||$\\
    $\alpha = \frac{1}{\sqrt{3}} $\\
    $x_0 = 0 $\\
    $x_3=x_2=x_1=F(x_0) = 0$\\
    $||x_3 - a|| \left( \frac{\alpha^3}{1-\alpha}||x_0 - x_1|| \right) = 0$

\end{flushleft}

\section{№4.8}
   $$ F: X \rightarrow Y$$\\
    $X = L_2[0;1], Y = L_2[0;1]$\\
    $F(x) = tx(t^2) = [r=t^2] = \sqrt{r}x(r) $\\
    $||F(x)||_{L_2[0;1]} = \left( \int_0^1 |r^{1\over2}x(r)|^2dr\right)^{1\over2} = \left( \int_0^1 |r||x(r)|^2dr\right)^{1\over2} \leq \left( \int_0^1 {max}_{r\in[0;1]} |r||x(r)|^2dr\right)^{1\over2} = \left( \int_0^1 |x(r)|^2dr\right)^{1\over2} =$ \\
    $= ||x||_{L_2[0;1]} \Rightarrow \text{ если } ||x||_{L_2[0;1]} < \sigma, \text{ то }  ||F(x)||_{L_2[0;1]} < \sigma \Rightarrow \text{F непрервына в точке }x_0$ \\
    $\varepsilon > 0, \text{покажем, что }\exists r(\varepsilon):$\\
    $\forall x(t), y(t) \in L_2[0;1]\text{ таких, что }||x-y|| = \left( \int_0^1 |x(t)-y(t)|^2dt\right)^{1\over2} < \sigma, \text{ выполняется:}$\\
    $||F(x)-F(y)|| = \left( \int_0^1 |F(x)-F(y)|^2dt\right)^{1\over2} = \left( \int_0^1 |r^{1\over2}(x(r)-y(r))|^2 dr\right)^{1\over2 }$\\
   $ |F(x)-F(y)| = |r^{1\over2}(x(r)-y(r))| \leq |x(r)-y(r)| \Rightarrow \varepsilon \leq \sigma \Rightarrow \text{F равномерно непрерывна} $\\
   $ ||F(x)-F(y)||_{L_2[0;1]} \leq c||x-y||_{L_2[0;1]} $\\
   $ ||F(x)-F(y)||_{L_2[0;1]} = \left( \int_0^1 |r^{1\over2}(x(r)-y(r))|^2 dr\right)^{1\over2} = \left( \int_0^1 |r||x(r)-y(r)|^2 dr\right)^{1\over2} \leq $\\ 
   $ \leq \left(\int_0^1 {max}_{z\in[0;1]}|r||x(r)-y(r)|^2 dr\right)^{1\over2} = ||x-y||, $\\
   $ \text{где с=1, таким образом F удовелтворяет условию Липшица}$


\end{document}